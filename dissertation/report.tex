\documentclass[10pt, onecolumn, letterpaper]{article}

\usepackage[utf8x]{inputenc}
\usepackage[T1]{fontenc}
\usepackage[a4paper, margin=2cm]{geometry}
\usepackage{amsmath}
\usepackage{graphicx}
\usepackage[colorlinks=true, allcolors=black]{hyperref}
\usepackage{natbib}
\usepackage{subcaption}
\usepackage{subfiles}
\usepackage{titlesec}
\usepackage{csvsimple}
\usepackage{fancyvrb}
\usepackage{multicol}
\usepackage{float}

\newcommand{\charactercount}[1]{\immediate\write18{expr `texcount -1 -sum -merge #1.tex` + `texcount -1 -sum -merge -char #1.tex` - 1 > chars.txt}\input{chars.txt}characters}

\graphicspath{{figures/}}
\bibliographystyle{apalike}
\titleformat*{\section}{\centering\small}
\titleformat*{\subsection}{\itshape\small}
\renewcommand{\thesection}{\Roman{section} }
\renewcommand{\thesubsection}{\alph{subsection}}
\renewcommand{\bibsection}{\null \\ \centerline{\small{REFERENCES}}}
\raggedbottom
\raggedcolumns

\title{
    \vspace{-0.5cm}
    \Large\textbf{Type-Extensible Object Notation: JSON with Syntax for Types} \\
    \vspace{0.5cm}
    \large\textbf{Master Thesis in Information Technology}
}
\author{
    Thor Wessel Lindberg \\
    \small{Department of Digital Design} \\
    \small{IT-University of Copenhagen, Denmark} \\
    \small{\href{mailto:mawl@itu.dk}{mawl@itu.dk}}
    \and
    Dr. Jørgen Staunstrup \\
    \small{Department of Computer Science}  \\
    \small{IT-University of Copenhagen, Denmark} \\
    \small{\href{mailto:jst@itu.dk}{jst@itu.dk}}
}
\date{}

\begin{document}

\color{red}
\centerline{ \Large This document contains \charactercount{report} (incl. spaces) }
\color{black}

\maketitle
\subfile{sections/abstract}

\begin{multicols}{2}
\subfile{sections/introduction}
\subfile{sections/case}
\subfile{sections/vocabulary}
\subfile{sections/relatedwork}
\subfile{sections/experiment}
\subfile{sections/results}
\subfile{sections/discussion}
\subfile{sections/conclusion}
\subfile{sections/futurework}
\bibliography{bibliography}
\end{multicols}

\end{document}

% levels of architecture: system -> data -> information. Benjamin says ETL is the data engineering component of data architecture
% 2x distributed computing, 1x data transmission, 1x object serialisation
%\subfile{sections/parsing}
% 2x object deserialisation + validation, 1x defensive mechanisms, 1x debugging     (library: JSON) (interviews: grounded theory)
% data persistence, interoperability concerns                                       (library: XML)
% typescript, design philosophy                                                     (tests: validation)
% trans-compilation, full backwards-compatibility                                   (library: TXON)

% related work: serialisation performance
% experiment setup: API calls, defensive code, JSON/XML specification, validation script, extensibility transpiler