\documentclass[10pt, twocolumn, letterpaper]{article}

%% Packages

\usepackage[utf8x]{inputenc}
\usepackage[T1]{fontenc}
\usepackage[a4paper, margin=2cm]{geometry}
\usepackage{amsmath}
\usepackage{graphicx}
\usepackage[colorinlistoftodos]{todonotes}
\usepackage[colorlinks=true, allcolors=black]{hyperref}
\usepackage{natbib}
\usepackage{subcaption}
\usepackage{subfiles}
\usepackage{titlesec}
\usepackage[english]{babel}
\usepackage{csvsimple}
\usepackage{authblk}

%% Customisation

\graphicspath{{figures/}}
\bibliographystyle{apalike}
\titleformat*{\section}{\centering\small}
\titleformat*{\subsection}{\textit\small}
\renewcommand{\thesection}{\Roman{section}} 
\renewcommand{\thesubsection}{\alph{subsection}}
\renewcommand{\bibsection}{\null \\ \centerline{\small{REFERENCES}}}
\raggedbottom

%% Title

\title{
    \vspace{-0.5cm}
    \huge \textbf{Modelling Architectural Properties of}  \\
    \huge \textbf{Plaintext and Binary Object Serialisation}  \\
}
\author{Thor Wessel Lindberg (17858, mawl@itu.dk)}
\date{}

%% Content

\begin{document}

\maketitle

\subfile{sections/preface}

\section{INTRODUCTION}
\subfile{sections/introduction}

\section{DISTRIBUTED COMPUTING} % layers, OSI model
\subfile{sections/distributed}

\section{DATA TRANSMISSION} % REST APIs
\subfile{sections/transmission}

\section{OBJECT SERIALISATION} % why, state, formats, advantages and disadvantages
\subfile{sections/serialisation}

\section{SOFTWARE ARCHITECTURE} % backend, frontend, distribution, developers
\subfile{sections/architecture}

\section{IMPLEMENTATION CASE} % plaintext -> human readable
\subfile{sections/implementation}

\section{RELATED WORK} % object serialisation, format/performance/size comparison
\subfile{sections/relatedwork}

\section{HUMAN-READABILITY VS TYPE SAFETY} % validity and type safety
\subfile{sections/comparison}

\section{DISCUSSION}
\subfile{sections/discussion}

\section{CONCLUSION}
\subfile{sections/conclusion}

\section{FUTURE WORK}
\subfile{sections/futurework}

\bibliography{bibliography}

\end{document}
