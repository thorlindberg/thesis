\documentclass[10pt, twocolumn, letterpaper]{article}

\usepackage[utf8x]{inputenc}
\usepackage[T1]{fontenc}
\usepackage[a4paper, margin=2cm]{geometry}
\usepackage{amsmath}
\usepackage{graphicx}
\usepackage[colorlinks=true, allcolors=black]{hyperref}
\usepackage{natbib}
\usepackage{subcaption}
\usepackage{subfiles}
\usepackage{titlesec}
\usepackage{csvsimple}
\usepackage{fancyvrb}
\usepackage{float}

\graphicspath{{figures/}}
\bibliographystyle{apalike}
\titleformat*{\section}{\centering\small}
\titleformat*{\subsection}{\textit\small}
\renewcommand{\thesection}{\Roman{section}}
\renewcommand{\thesubsection}{\alph{subsection}}
\renewcommand{\bibsection}{\null \\ \centerline{\small{REFERENCES}}}
\raggedbottom

\title{
    \vspace{-0.5cm}
    \Large\textbf{Type-Extensible Object Notation: JSON with Syntax for Types} \\
    \vspace{0.5cm}
    \large\textbf{Master Thesis in Information Technology}
}
\author{
    Thor Wessel Lindberg \\
    \small{Department of Digital Design} \\
    \small{IT-University of Copenhagen, Denmark} \\
    \small{\href{mailto:mawl@itu.dk}{mawl@itu.dk}}
    \and
    Dr. Jørgen Staunstrup \\
    \small{Department of Computer Science}  \\
    \small{IT-University of Copenhagen, Denmark} \\
    \small{\href{mailto:jst@itu.dk}{jst@itu.dk}}
}
\date{}

\begin{document}

\maketitle

\subfile{sections/preface}

\section{INTRODUCTION}
\subfile{sections/introduction}

\section{DISTRIBUTED COMPUTING} % layers, OSI model
\subfile{sections/distributed}

\section{DATA TRANSMISSION} % REST APIs
\subfile{sections/transmission}

\section{OBJECT SERIALISATION} % why, state, formats, advantages and disadvantages
\subfile{sections/serialisation}

\section{RELATED WORK} % object serialisation, format/performance/size comparison
\subfile{sections/relatedwork}

\section{FORMATS} % validity and type safety
\subfile{sections/comparison}

\section{DISCUSSION}
\subfile{sections/discussion}

\section{CONCLUSION}
\subfile{sections/conclusion}

\section{FUTURE WORK}
\subfile{sections/futurework}

\bibliography{bibliography}

\end{document}
