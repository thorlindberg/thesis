\documentclass[../report.tex]{subfiles}

\begin{document}

The OSI Model (Open Systems Interconnection Model) is a conceptual framework used to describe the functions of a networking system. The OSI model characterizes computing functions into a universal set of rules and requirements in order to support interoperability between different products and software. In the OSI reference model, the communications between a computing system are split into seven different abstraction layers: Physical, Data Link, Network, Transport, Session, Presentation, and Application.

Created at a time when network computing was in its infancy, the OSI was published in 1984 by the International Organization for Standardization (ISO). Though it does not always map directly to specific systems, the OSI Model is still used today as a means to describe Network Architecture.

\subsection{The 7 Layers of the OSI Model}

\textbf{Physical Layer.} The lowest layer of the OSI Model is concerned with electrically or optically transmitting raw unstructured data bits across the network from the physical layer of the sending device to the physical layer of the receiving device. It can include specifications such as voltages, pin layout, cabling, and radio frequencies. At the physical layer, one might find “physical” resources such as network hubs, cabling, repeaters, network adapters or modems.

\textbf{Data Link Layer.} At the data link layer, directly connected nodes are used to perform node-to-node data transfer where data is packaged into frames. The data link layer also corrects errors that may have occurred at the physical layer. The data link layer encompasses two sub-layers of its own. The first, media access control (MAC), provides flow control and multiplexing for device transmissions over a network. The second, the logical link control (LLC), provides flow and error control over the physical medium as well as identifies line protocols.

\textbf{Network Layer.} The network layer is responsible for receiving frames from the data link layer, and delivering them to their intended destinations among based on the addresses contained inside the frame. The network layer finds the destination by using logical addresses, such as IP (internet protocol). At this layer, routers are a crucial component used to quite literally route information where it needs to go between networks.

\textbf{Transport Layer.} The transport layer manages the delivery and error checking of data packets. It regulates the size, sequencing, and ultimately the transfer of data between systems and hosts. One of the most common examples of the transport layer is TCP or the Transmission Control Protocol.

\textbf{Session Layer.} The session layer controls the conversations between different computers. A session or connection between machines is set up, managed, and termined at layer 5. Session layer services also include authentication and reconnections.

\textbf{Presentation Layer.} The presentation layer formats or translates data for the application layer based on the syntax or semantics that the application accepts. Because of this, it at times also called the syntax layer. This layer can also handle the encryption and decryption required by the application layer.

\textbf{Application Layer.} At this layer, both the end user and the application layer interact directly with the software application. This layer sees network services provided to end-user applications such as a web browser or Office 365. The application layer identifies communication partners, resource availability, and synchronizes communication.

% Formats, Protocols, Blockchain?

\end{document}
