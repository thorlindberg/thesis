\documentclass[../report.tex]{subfiles}

\begin{document}

\section{INTRODUCTION}

As mobile and internet of things (IoT) devices continue to dominate the computing space, software development increasingly centers around clients receiving transmitted data. This system of connected devices is known as distributed computing, consisting of multiple heterogeneous or homogeneous bodies.

Heterogeneity indicates a inconsistency in architecture, language, data representation, instruction set and hardware capabilities. As a result, the data transmitted is serialised and deserialised to a homogeneous format shareable across heterogeneous bodies. This dependence on serialisation motivates the comparison of different formats for object serialisation. 

Existing research has already compared formats on syntax, performance, and efficiency, but has not addressed the architectural differences. As serialisation does not exist in a vacuum, the shortcomings of a format has to be guarded against through the development of defensive mechanisms.

The contribution of this paper is to provide an architectural comparison of object serialisations formats. This is accomplished by comparing the formats themselves, then modelling software architectures for object serialisation.

% In the year 2021, ?\% of web traffic and ?\% of content consumption was attributed to mobile devices. The rise of the smartphone and "Internet of Things" (IOT) devices has shifted software markets and the attention of developers away from the desktop. In combination with the expansion of social media, most software now acts as content consumption clients; requesting and receiving serialised data through REST APIs, then parsing and presenting it to users. Serialisation of data is typically handled on the backend (server), while the frontend (application) handles deserialisation.

% The contribution of this paper is to provide the software development community a comparative assessment of the trade-off between usability and type safety between human-readable and binary data serialisation formats. The ubiquity of data transmission across heterogeneous languages, formats, and systems in a distributed computing environment, necessitates an evaluation of data serialisation formats for software implementation. Human-readable formats such as XML and JSON are comparatively analysed to binary formats such as Google's Protocol Buffers. Another alternative is a full stack solution that handles the entire stack of data transmission and serialisation, to guard against type errors etc.

\end{document}