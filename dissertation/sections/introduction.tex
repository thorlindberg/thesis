\documentclass[../report.tex]{subfiles}

\begin{document}

In the year 2021, ?\% of web traffic and ?\% of content consumption was attributed to mobile devices. The rise of the smartphone and "Internet of Things" (IOT) devices has shifted software markets and the attention of developers away from the desktop. In combination with the expansion of social media, most software now acts as content consumption clients; requesting and receiving serialised data through REST APIs, then parsing and presenting it to users. Serialisation of data is typically handled on the backend (server), while the frontend (application) handles deserialisation.

The contribution of this paper is to provide the software development community a comparative assessment of the trade-off between usability and type safety between human-readable and binary data serialisation formats. The ubiquity of data transmission across heterogeneous languages, formats, and systems in a distributed computing environment, necessitates an evaluation of data serialisation formats for software implementation. Human-readable formats such as XML and JSON are comparatively analysed to binary formats such as ProtoBuf and Thrift.

\end{document}