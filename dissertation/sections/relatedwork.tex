\documentclass[../report.tex]{subfiles}

\begin{document}

Previous research provides a baseline for building upon existing knowledge through this project. This research typically focuses on documenting the object serialisation process or comparing serialisation formats in terms of features, efficiency, performance, file size, and programming language support. \\

\cite{kazuaki2011survey} surveyed object serialisation techniques, concluding that each technique had its advantages and disadvantages in the context it was applied. \\

\cite{eriksson2011comparison} compared the features of two plaintext object serialisation formats: JavaScript Object Notation (JSON) and YAML, then determined their efficiency by measuring performance and data storage size. \\

\cite{goff2001xmlserialization} document object serialisation processes with the eXtensible Markup Language (XML) format, assessing its implementation in heterogeneous distributed systems. \\

\cite{sumaray2012efficiency} ... \\

\cite{tauro2012binary} document implementation techniques for binary object serialisation in the programming languages: C++, Java and .NET. They conclude that binary serialisation is memory and bandwidth efficient. \\

\cite{vanura2018performance} ...

\end{document}
