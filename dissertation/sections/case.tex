\documentclass[../report.tex]{subfiles}

\begin{document}

\section{IMPLEMENTATION CASE} \label{sec:case}

In this project I've chosen to collaborate with a company that specialises in native mobile application development. Their identity is kept anonymous, so rather than include confidential data or code samples, I have chosen to derive generic examples from the material they have provided me.

This company holds a unique perspective relative to the landscape of software development in Copenhagen, where return on investment (ROI) in my optics is valued above quality. Rather than take the typical multi-platform approach, using a platform-neutral framework like \textit{React Native}, they maintain independent development teams for each platform, and they work exclusively with native code. They maintain an Android team utilising \textit{Flutter} and an iOS team utilising \textit{Swift}. This nets them hardware efficiency and performance advantages, at the cost of operating and aligning two parallel developer teams working on the same projects.

\subsection{Organisation structure and hierarchy}

...

% Stakeholders, responsibilities, and UML diagrams

\subsection{Perspectives on serialised data}

This section presents the personas derived from interviews with each development team at the company, for the purpose of taking value-oriented design decisions derived from their perspectives. As these employees are coworkers, their perspectives are grounded in shared experiences, yet their unique deviations highlight how serialised data has varying implications on work practices. \\

\begin{figure*}
\def\arraystretch{1.5}
\centering
\begin{tabular}{|p{0.3\linewidth}|p{0.5\linewidth}|}
\hline
Feature & Description \\
\hline
... & ... \\ 
... & ... \\ 
... & ... \\ 
\hline
\end{tabular}
\vspace{0.5cm} \\
\begin{tabular}{|p{0.3\linewidth}|p{0.5\linewidth}|}
\hline
Feature & Description \\
\hline
... & ... \\ 
... & ... \\ 
... & ... \\ 
\hline
\end{tabular}
\caption{Personas.}
\label{fig:personas}
\end{figure*}

\end{document}