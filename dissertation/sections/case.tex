\documentclass[../report.tex]{subfiles}

\begin{document}

\section{IMPLEMENTATION CASE} \label{sec:case}

% Implementation case
%% Stakeholders
%% UML diagram
%% Personas

In this project I've chosen to collaborate with a company that specialises in native mobile application development. Their identity is kept anonymous, so rather than include confidential data or code samples, I have chosen to derive generic examples from the material they have provided me.

This company holds a unique perspective relative to the landscape of software development in Copenhagen, where return on investment (ROI) in my optics is valued above quality. Rather than take the typical multi-platform approach, using a platform-neutral framework like React Native, they maintain independent development teams for each platform, and they work exclusively with native code. In practice they maintain an Android team utilising \textit{Flutter}, and an iOS team utilising \textit{Swift}. This results in efficiency and performance advantages, at the cost of operating and aligning two parallel teams rather than one.

\subsection{Perspectives on serialised data}

This section presents the personas derived quantitatively, and relates them to the qualitative interviews conducted... \\

\begin{figure}[H]
\def\arraystretch{1.5}
\centering
\begin{tabular}{|p{0.3\linewidth}|p{0.5\linewidth}|}
\hline
Feature & Description \\
\hline
... & Yes \\ 
... & No \\ 
... & Yes \\ 
\hline
\end{tabular}
\end{figure}

\begin{figure}[H]
\def\arraystretch{1.5}
\centering
\begin{tabular}{|p{0.3\linewidth}|p{0.5\linewidth}|}
\hline
Feature & Description \\
\hline
... & Yes \\ 
... & No \\ 
... & Yes \\ 
\hline
\end{tabular}
\end{figure}

\begin{figure}[H]
\def\arraystretch{1.5}
\centering
\begin{tabular}{|p{0.3\linewidth}|p{0.5\linewidth}|}
\hline
Feature & Description \\
\hline
... & Yes \\ 
... & No \\ 
... & Yes \\ 
\hline
\end{tabular}
\end{figure}

\end{document}